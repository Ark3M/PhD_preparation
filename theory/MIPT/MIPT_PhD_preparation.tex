\documentclass{article}
\usepackage{packages}
\usepackage{environments}
\usepackage{commands}


\begin{document}

\begin{titlepage}
	\centering
	\vspace*{\fill} % Vertically center the text
	\Huge{Подготовка к поступлению в аспирантуру 2023} \\
	\Large{\textbf{МФТИ}} \\
	\Large{1.2 Компьютерные науки и информатика (ФПМИ)} \\
	\vspace{2em} % Add some vertical space between the title and author
	\Large\textbf{Матвеев Артём}
	\vspace*{\fill} % Vertically center the text
	\newpage
\end{titlepage}
\setcounter{page}{2}

\part*{Математический анализ}

\section*{Пределы последовательности. Критерий Коши. Существование предела у монотонно возрастающей ограниченной сверху последовательности. Теорема Больцано-Вейерштрассе о существовании сходящейся подпоследовательности у ограниченной последовательности}

\subsection*{Пределы последовательности}

Назовём окрестностью точки $a \in \mathbb R$ любой интервал, содержащий эту точку, и будем обозначать её как $\mathrm{U}(a)$.

$\mathrm{U}_{\epsilon}(a) = (a - \epsilon; a + \epsilon)$ -- симметричная $\epsilon$-окрестность.

$\mathrm{U}'_{\epsilon}(a) = (a - \epsilon; a) \cup (a; a + \epsilon) = \mathrm{U}^0_{\epsilon} = \mathrm{U}_{\epsilon} \setminus \{a\}$ -- проколотая.

Будем говорить, что числовая последовательность $\{x_n\}$ имеет \textbf{предел, равный} $a$, если:
\begin{equation}\label{eq:lim1}
	\lim_{n \to \infty} x_n = a \Leftrightarrow \forall \mathrm{U}_{\epsilon}(a)\ \exists N (\mathrm{U}_{\epsilon}(a)) \colon \forall n \ge N \Rightarrow x_n \in \mathrm{U}_{\epsilon}(a)
\end{equation}

\begin{equation}\label{eq:lim2}
	\lim_{n \to \infty} x_n = a \Leftrightarrow \forall \epsilon > 0\ \exists N (\epsilon) \colon \forall n \ge N \Rightarrow \mathopen| x_n - a \mathclose| < \epsilon
\end{equation}

Определения \eqref{eq:lim1} и \eqref{eq:lim2} эквивалентны. 

Если существует $\lim\limits_{n \to \infty} x_n = a < \infty$, то $\{x_n\}$ -- \textbf{сходящаяся} последовательность. Иначе, если не существует  $\lim\limits_{n \to \infty} x_n$ или $\lim\limits_{n \to \infty} x_n = \infty$, то $\{x_n\}$ -- \textbf{расходящаяся}.
\pagebreak




\part*{Линейная алгебра \blfootnote{Все ответы составлены на основе \href{https://github.com/DimaTrushin}{материалов} Димы Трушина.}}

\section*{1. Понятие линейного пространства. Определение линейно зависимости и независимости векторов. Размерность линейного пространства. Базис, координаты вектора, формулы преобразования координат при переходе от одного базиса к другому}

\subsection*{Понятие линейного пространства}

Следующее определение -- это пример определения с контекстом.
Это означает, что прежде, чем его дать, мы должны зафиксировать некоторую информацию, которая необходима для нашего определения и без этой информации оно -- бессмысленный мусор.
У определения векторного (линейного) пространства в качестве такого контекста выступает некоторое поле $F$.
Это значит, что пока мы не зафиксировали какое-то поле, мы не можем говорить о векторных пространствах над полем $F$, а <<просто векторных пространств>> без указания какого-либо поля не существует.

\begin{definition*}
	\label{def::VectorSpace}
	Пусть $F$ -- некоторое фиксированное поле.
	Тогда векторное пространство над полем $F$ -- это следующий набор данных $(V, +, \cdot)$, где
	\begin{itemize}
		\item $V$ -- множество.
		Элементы этого множества будут называться векторами.
		
		\item $+\colon V \times V \to V$ -- бинарная операция, то есть правило, действующее так: $(v,u)\mapsto v + u$, где $u,v \in V$.
		
		\item $\cdot \colon F \times V \to V$ -- бинарная операция, то есть правило, действующее так: $(\alpha, v)\mapsto \alpha v$, где $\alpha \in F$ и $v\in V$.
	\end{itemize}
	При этом эти данные удовлетворяют следующим $8$ аксиомам:
	\begin{enumerate}
		\item {\bf Ассоциативность сложения} Для любых векторов $u,v,w\in V$ верно $(u+v) + w = u + (v+w)$.
		
		\item {\bf Существование нулевого вектора} Существует такой вектор $0\in V$, что для любого $v\in V$ выполнено $0 + v = v + 0 = v$.
		
		\item {\bf Существование противоположного вектора} Для любого вектора $v\in V$ существует вектор $-v\in V$ такой, что $v + (-v) = (-v) + v = 0$.
		
		\item {\bf Коммутативность сложения} Для любых векторов $u,v \in V$ верно $u + v = v + u$.
		
		\item {\bf Согласованность умножения со сложением векторов} Для любого числа $\alpha \in F$ и любых векторов $u,v \in V$ верно $\alpha(v + u) = \alpha v + \alpha u$.
		
		\item {\bf Согласованность умножения со сложением чисел} Для любых чисел $\alpha, \beta\in F$ и любого вектора $v\in V$ верно $(\alpha + \beta)v = \alpha v + \beta v$.
		
		\item {\bf Согласованность умножения с умножением чисел} Для любых чисел $\alpha,\beta\in F$ и любого вектора $v\in V$ верно $(\alpha\beta)v = \alpha(\beta v)$.
		
		\item {\bf Нетривиальность} Для любого $v\in V$ верно $1 v = v$.%
		\footnote{Здесь $1\in F$.}
	\end{enumerate}
\end{definition*}

\paragraph{Примеры}

\begin{enumerate}
	\item Поле $F$ (или кто больше привык к вещественным числам $\mathbb R$) является векторным пространством над $F$ (соответственно над $\mathbb R$).
	
	\item Более обще, множество вектор-столбцов $F^n$ является векторным пространством над $F$.
	
	\item Множество матриц $\operatorname{M}_{m\,n}(F)$ является векторным пространством над $F$.
	
	\item Пусть $X$ -- произвольное множество, тогда множество функций $F(X, F) = \{f\colon X\to F\}$ является векторным пространством над $F$.
	Надо лишь объяснить как складывать функции и умножать на элементы $F$.
	Операции поточечные, пусть $f,g\colon X\to F$, тогда функция $(f+g)\colon X\to F$ действует по правилу $(f+g)(x) = f(x) + g(x)$.
	Если $\alpha \in F$, то функция $(\alpha f)\colon X\to F$ действует по правилу $(\alpha f)(x) = \alpha f(x)$.
	
	\item Множество многочленов $F[x] = \{a_0+a_1x + \ldots + a_n x^n\mid a_i \in F,\,n\in \mathbb Z_{\geqslant 0}\}$.
	Тут надо обратить внимание, что мы подразумеваем под многочленом.
	Для нас многочлен -- это НЕ функция, многочлен -- это картинка вида $a_0 + a_1 x + \ldots + a_n x^n$.%
	\footnote{Для любителей формализма, можете считать, что многочлен -- это конечная последовательность элементов $F$ вида $(a_0,\ldots,a_n)$, но длина последовательности может быть любой, включая нулевую.}
	Складываются и умножаются эти картинки по одинаковым правилам.
	Важно, что две такие картинки равны тогда и только тогда, когда у них равные коэффициенты.
	Множество всех многочленов $F[x]$ является векторным пространством над $F$.
\end{enumerate}

\subsection*{Определение линейной зависимости и линейной независимости векторов}

\subsubsection*{Линейные комбинации}

\textbf{Мотивация}. Пусть у нас есть векторное пространство $V$ над полем $F$.
Давайте поймем, а что вообще с ним можно делать?
Во-первых, $V$ -- это множество.
Значит из него можно брать элементы.
Во-вторых, там есть операция умножения на числа, то есть любой вектор можно умножить на какое-то число.
В-третьих, вектора можно складывать.
Все это означает, что все что можно делать с векторным пространством, это набрать каких-то векторов из него $v_1,\ldots,v_n$ и написать выражение вида $\alpha_1 v_1 + \ldots + \alpha_n v_n$, для произвольных $\alpha_i\in F$.
Это выражение будет задавать нам какой-то вектор из $V$.
Как мы видим, особенно не разбежишься с разнообразием действий.
Однако, важно, что с помощью подобных выражений можно вытащить абсолютно всю информацию из векторных пространств, которую только возможно.
Именно поэтому все наше внимание будет посвящено выражениям такого вида, так как из них получится узнать все, что только можно про векторные пространства.

\begin{definition*}
	Пусть $V$ -- некоторое векторное пространство над полем $F$ и пусть $v_1,\ldots,v_n\in V$ -- некоторый набор векторов.
	Тогда выражение вида $\alpha_1 v_1 +\ldots + \alpha_n v_n$, где $\alpha_i\in F$, называется линейной комбинацией $v_1,\ldots,v_n$.
	Линейная комбинация называется тривиальной, если все $\alpha_i = 0$.
	В противном случае она называется нетривиальной.
\end{definition*}

\begin{definition*}
	Вектора $v_1,\ldots,v_n\in V$ называются линейно зависимыми, если существует их нетривиальная линейная комбинация равная нулю, то есть для каких-то $\alpha_i\in F$ (так что хотя бы один не равен нулю) выражение $\alpha_1 v_1+\ldots + \alpha_n v_n = 0$.
	Подчеркнем, что вектора линейно независимы, если из равенства $\alpha_1 v_1 + \ldots + \alpha_n v_n = 0$ следует, что все $\alpha_i = 0$.
\end{definition*}

\paragraph{Примеры}

\begin{enumerate}
	\item Вектор $0$ всегда линейно зависим.
	
	\item Вектор $v\in V$ линейно зависим тогда и только тогда, когда он равен нулю.
	
	\item Вектора $v_1, v_2 \in V$ линейно зависимы тогда и только тогда, когда они пропорциональны (то есть один из них равен другому умноженному на элемент поля).
\end{enumerate}

Заметим, что если множество векторов $v_1,\ldots, v_k$ линейно независимо, то и любое его подмножество тоже линейно независимо.
Потому интересно не уменьшать, а увеличивать линейно независимые подмножества векторов.
Линейно независимое множество векторов $v_1,\ldots, v_k$ называется максимальным, если при добавлении к нему любого вектора оно становится линейно зависимым.
\pagebreak

\section*{2. Матрицы и действия над ними. Детерминант квадратной матрицы. Ранг матрицы. Эквивалентность его двух определений в терминах линейной независимости строк (или столбцов) матрицы и в терминах неравенства нулю миноров}

Матрица -- это прямоугольная таблица чисел
\begin{equation*}
A=
\begin{pmatrix}
	a_{11} & \ldots & a_{1n} \\
	\vdots & \ddots & \vdots \\
	a_{m1} & \ldots & a_{mn}
\end{pmatrix}, \text{ где } a_{ij} \in \mathbb{R}
\end{equation*}

Множество всех матриц с $m$ строками и $n$ столбцами обозначается $\MatrixDim{m}{n}$. Множество квадратных матриц размера $n$ будем обозначать $\Matrix{n}$. Матрицы с одним столбцом или одной строкой называются векторами (вектор-столбцами и вектор-строками соответственно). Множество всех векторов с $n$ координатами обозначается $\Vector{n}$. По умолчанию будем считать, что наши вектора -- это вектор-столбцы.

\subsection*{Операции с матрицами}

\subsubsection*{1. Сложение} Пусть $A, B \in \MatrixDim{m}{n}$. Тогда сумма $A + B$ определяется покомпонентно, т.е. $C = A + B$ и $c_{ij} = a_{ij} + b_{ij}$. В явном матричном виде:

\[
\begin{pmatrix}
	a_{11} & \ldots & a_{1n} \\
	\vdots & \ddots & \vdots \\
	a_{m1} & \ldots & a_{mn}
\end{pmatrix}
+
\begin{pmatrix}
	b_{11} & \ldots & b_{1n} \\
	\vdots & \ddots & \vdots \\
	b_{m1} & \ldots & b_{mn}
\end{pmatrix}
=
\begin{pmatrix}
	a_{11}+b_{11} & \ldots & a_{1n}+b_{1n} \\
	\vdots & \ddots & \vdots \\
	a_{m1}+b_{m1} & \ldots & a_{mn}+b_{mn}
\end{pmatrix}
\]

Складывать можно только матрицы одинакового размера.

\subsubsection*{2. Умножение на скаляр} Если $\lambda\in \mathbb R$ и $A\in \MatrixDim{m}{n}$, то $\lambda A$ определяется так: $\lambda A = C$, где $c_{ij} = \lambda a_{ij}$ или

\[
\lambda
\begin{pmatrix}
	a_{11}&\ldots& a_{1n}\\
	\vdots&\ddots&\vdots\\
	a_{m1}& \ldots &a_{mn}
\end{pmatrix}
=
\begin{pmatrix}
	\lambda a_{11}&\ldots& \lambda a_{1n}\\
	\vdots&\ddots&\vdots\\
	\lambda a_{m1}& \ldots &\lambda a_{mn}
\end{pmatrix}
\]

\subsubsection*{3. Умножение матриц} Пусть $A\in\MatrixDim{m}{n}$ и $B\in\MatrixDim{n}{k}$, то произведение $AB\in\MatrixDim{m}{k}$ определяется так: $AB = C$, где $c_{ij} = \sum_{t=1}^n a_{it}b_{tj}$ или

\[
\begin{pmatrix}
	a_{11}&\ldots& a_{1n}\\
	\vdots&\ddots&\vdots\\
	a_{m1}& \ldots &a_{mn}
\end{pmatrix}
\begin{pmatrix}
	b_{11}&\ldots& b_{1k}\\
	\vdots&\ddots&\vdots\\
	b_{n1}& \ldots &b_{nk}
\end{pmatrix}
=
\begin{pmatrix}
	\sum_{t=1}^n a_{1t}b_{t1}&\ldots& \sum_{t=1}^n a_{1t}b_{tk}\\
	\vdots&\ddots&\vdots\\
	\sum_{t=1}^n a_{mt}b_{t1}& \ldots &\sum_{t=1}^n a_{mt}b_{tk}
\end{pmatrix}
\]

На умножение матриц можно смотреть следующим образом. Чтобы получить коэффициент $c_{ij}$ надо, из матрицы $A$ взять $i$-ю строку (она имеет длину $n$), а из матрицы $B$ взять $j$-ый столбец (он тоже имеет длину $n$). Тогда их надо скалярно перемножить и результат подставить в $c_{ij}$.

\subsubsection*{4. Транспонирование} Пусть $A$ -- это матрица вида

\[
\begin{pmatrix}
	{a_{11}}&{\ldots}&{a_{1n}}\\
	{\vdots}&{\ddots}&{\vdots}\\
	{a_{m1}}&{\ldots}&{a_{mn}}\\
\end{pmatrix}\quad \text{или}\quad
\begin{pmatrix}
	{a_{11}}&{a_{12}}&{a_{13}}\\
	{a_{21}}&{a_{22}}&{a_{23}}
\end{pmatrix}\quad \text{или}\quad
\begin{pmatrix}
	{x_1}\\
	{x_2}\\
	{x_3}\\
\end{pmatrix}
\]

Определим транспонированную матрицу $A^t = (a'_{ij})$ так: $a'_{ij} = a_{ji}$. Наглядно, транспонированная матрица для приведенных выше

\[
\begin{pmatrix}
	{a_{11}}&{\ldots}&{a_{m1}}\\
	{\vdots}&{\ddots}&{\vdots}\\
	{a_{1n}}&{\ldots}&{a_{mn}}\\
\end{pmatrix}\quad\text{или}\quad
\begin{pmatrix}
	{a_{11}}&{a_{21}}\\
	{a_{12}}&{a_{22}}\\
	{a_{13}}&{a_{23}}\\
\end{pmatrix}\quad \text{или}\quad
\begin{pmatrix}
	{x_1}&{x_2}&{x_3}\\
\end{pmatrix}
\]


\subsubsection*{Свойства операций}

\begin{enumerate}
	\item {\bf Ассоциативность сложения} $(A + B) + C = A + (B + C)$ для любых $A,B,C\in \MatrixDim{m}{n}$
	\item {\bf Существование нейтрального элемента для сложения} Существует единственная матрица $0\in\MatrixDim{m}{n}$ обладающая следующим свойством $A + 0 = 0 + A = A$ для всех $A\in\MatrixDim{m}{n}$. Такая матрица целиком заполнена нулями.
	
	\item {\bf Коммутативность сложения} $A + B = B + A$ для любых $A,B\in\MatrixDim{m}{n}$.
	
	\item {\bf Наличие обратного по сложению} Для любой матрицы $A\in\MatrixDim{m}{n}$ существует матрица $-A$ такая, что $A + (-A) = (-A) + A = 0$. Такая матрица единственная и состоит из элементов $-a_{ij}$.
	
	\item {\bf Ассоциативность умножения} Для любых матриц $A\in\MatrixDim{m}{n}$, $B\in\MatrixDim{n}{k}$ и $C\in\MatrixDim{k}{t}$ верно $(AB)C = A(BC)$.
	
	\item {\bf Существование нейтрального элемента для умножения} Для каждого $k$ существует единственная матрица $E\in\Matrix{k}$ такая, что для любой $A\in\MatrixDim{m}{n}$ верно $E A = A E = A$. У такой матрицы $E_{ii} = 1$, а $E_{ij} = 0$. Когда нет путаницы матрицу $E$ обозначают через $1$.
	
	\item {\bf Дистрибутивность умножения относительно сложения} Для любых матриц $A,B\in\MatrixDim{m}{n}$ и $C\in\MatrixDim{n}{k}$ верно $(A + B)C = AC + B C$. Аналогично, для любых $A\in\MatrixDim{m}{n}$ и $B,C\in\MatrixDim{n}{k}$ верно $A(B+C) = AB + AC$.
	
	\item {\bf Умножение на числа ассоциативно} Для любых $\lambda,\mu \in\mathbb R$ и любой матрицы $A\in\MatrixDim{m}{n}$ верно $\lambda(\mu A) = (\lambda \mu) A$. Аналогично для любого $\lambda \in \mathbb R$ и любых $A\in\MatrixDim{m}{n}$ и $B\in \MatrixDim{n}{k}$ верно $\lambda(AB) = (\lambda A) B$.
	
	\item {\bf Умножение на числа дистрибутивно относительно сложения матриц и сложения чисел} Для любых $\lambda,\mu\in \mathbb R$ и $A\in \MatrixDim{m}{n}$ верно $(\lambda + \mu)A = \lambda A +\mu A$. Аналогично, для любого $\lambda\in\mathbb R$ и $A,B\in\MatrixDim{m}{n}$ верно $\lambda(A+B) = \lambda A + \lambda B$.
	
	\item {\bf Умножение на скаляр нетривиально} Если $1\in\mathbb R$, то для любой матрицы $A\in \MatrixDim{m}{n}$ верно $1 A = A$.
	
	\item {\bf Умножение на скаляр согласовано с умножением матриц} Для любого $\lambda \in \mathbb R$ и любых $A\in\MatrixDim{m}{n}$ и $B\in\MatrixDim{n}{k}$ верно $\lambda(AB) = (\lambda A)B = A (\lambda B)$.
\end{enumerate}


\subsection*{Детерминант}

Детерминант (или определитель) существует только для квадратных матриц. Классическое определение через перестановки. Пусть $A\in\Matrix{n}$ положим

\[
\det A = \sum_{\sigma\in\Sym{n}} \sgn(\sigma) a_{1\sigma(1)}\cdot \ldots \cdot a_{n\sigma(n)}
\]

Это и есть определитель матрицы $A$. Давайте поясним структуру формулы. Внутри стоят произведения вида $a_{1\sigma(1)}\cdot \ldots \cdot a_{n\sigma(n)}$, где первый индекс идет по-порядку, то есть мы выбираем по элементу из каждой строки. Второй индекс имеет вид $\sigma(i)$, где $\sigma$ -- перестановка, то есть эти индексы тоже не повторяются, то есть мы выбираем элементы так, что все они из разных строк и столбцов. После этого элементы перемножаются между собой, умножаются на знак $\sigma$ и все это радостно складывается в одну большую сумму из $n!$ слагаемых (таково количество перестановок).

Как это определение связано с ориентированным объемом. Думать про это надо так: пусть $A = (A_1\mid \ldots\mid A_n)\in\Matrix{n}$ составлена из столбцов $A_i\in\Vector{n}$. Тогда $\det A$ -- это ориентированный объем $n$-мерного параллелепипеда натянутого на вектора $A_1,\ldots,A_n$.

\paragraph{Примеры} 
\begin{enumerate}
	\item Если $A\in \Matrix{1} = \mathbb R$, то $\det A = A$.
	\item Если $A \in \Matrix{2}$ имеет вид $A = \left(\begin{smallmatrix}{a}&{b}\\{c}&{d}\end{smallmatrix}\right)$, то $\det A = ad - bc$. Графически: главная диагональ минус побочная.
	\item Если $A\in \Matrix{3}$ имеет вид $A = \left(\begin{smallmatrix}{a_{11}}&{a_{12}}&{a_{13}}\\{a_{21}}&{a_{22}}&{a_{23}}\\{a_{31}}&{a_{32}}&{a_{33}}\end{smallmatrix}\right)$, то определитель получается из $6$ слагаемых три из них с $+$ три с $-$. Графически слагаемые можно изобразить так:
	\[
	\det A = 
	+
	\left(
	\parbox{6pt}{
		\xymatrix@R=6pt@C=6pt{
			{}\ar@{-}[ddrr]&{}&{}\\
			{}&{}&{}\\
			{}&{}&{}\\
	}}
	\right) +
	\left(
	\parbox{6pt}{
		\xymatrix@R=6pt@C=6pt{
			{}&{}\ar@{-}[dr]&{}\\
			{}&{}&{}\ar@{-}[dll]\\
			{}\ar@{-}[uur]&{}&{}\\
	}}
	\right) +
	\left(
	\parbox{6pt}{
		\xymatrix@R=6pt@C=6pt{
			{}&{}&{}\ar@{-}[ddl]\\
			{}\ar@{-}[urr]&{}&{}\\
			{}&{}\ar@{-}[ul]&{}\\
	}}
	\right) -
	\left(
	\parbox{6pt}{
		\xymatrix@R=6pt@C=6pt{
			{}&{}&{}\ar@{-}[ddll]\\
			{}&{}&{}\\
			{}&{}&{}\\
	}}
	\right) -
	\left(
	\parbox{6pt}{
		\xymatrix@R=6pt@C=6pt{
			{}&{}\ar@{-}[ddr]&{}\\
			{}\ar@{-}[ur]&{}&{}\\
			{}&{}&{}\ar@{-}[ull]\\
	}}
	\right) -
	\left(
	\parbox{6pt}{
		\xymatrix@R=6pt@C=6pt{
			{}\ar@{-}[drr]&{}&{}\\
			{}&{}&{}\ar@{-}[dl]\\
			{}&{}\ar@{-}[uul]&{}\\
	}}
	\right) 
	\]
	Точная формула\footnote{Начиная с этого момента можно забыть честное определение. Оно почти никогда не нужно для использования, пригодятся лишь его свойства. Определитель никогда не считают по определению. Это слишком долго.}
	\[
	\det A = a_{11}a_{22}a_{33} + a_{12}a_{23}a_{31} + a_{13}a_{21}a_{32} - 
	a_{13}a_{22}a_{31} - a_{12}a_{21}a_{33} - a_{11}a_{23}a_{32}
	\]
\end{enumerate}

\subsubsection*{Свойства определителя}

Считать определитель по явной формуле весьма проблематично. Слишком уж много слагаемых. Потому по определению его можно вычислить лишь для очень специальных матриц. Для произвольных матриц используются некоторые полезные свойства, с помощью которых их определители сводятся к определителям специальных матриц.

Пусть $A\in \Matrix{n}$ -- матрица, тогда на нее можно смотреть как на набор из $n$ столбцов $A = (A_1|\ldots|A_n)$. Тогда определитель $\det (A)$ можно рассматривать, как функцию от столбцов матрицы $A$, то есть $\det (A) = \det(A_1|\ldots|A_n)$. Думаю таким образом, мы можем сформулировать следующие свойства:

\begin{enumerate}
	\item $\det(A_1|\ldots|A_i + A_i'|\ldots|A_n) = \det(A_1|\ldots|A_i|\ldots|A_n) + \det(A_1|\ldots|A_i'|\ldots|A_n)$. Например,
	\[
	\det
	\begin{pmatrix}
		{1}&{3}\\
		{2}&{7}
	\end{pmatrix}
	=
	\det
	\left(\left.
	\begin{pmatrix}
		{1}\\{2}
	\end{pmatrix}
	\right|
	\begin{pmatrix}
		{3}\\{7}
	\end{pmatrix}
	\right)
	=
	\det
	\left(\left.
	\begin{pmatrix}
		{1}\\{2}
	\end{pmatrix}
	\right|
	\begin{pmatrix}
		{1}\\{3}
	\end{pmatrix}
	+
	\begin{pmatrix}
		{2}\\{4}
	\end{pmatrix}
	\right)
	=
	\det
	\begin{pmatrix}
		{1}&{1}\\
		{2}&{3}
	\end{pmatrix}
	+
	\det
	\begin{pmatrix}
		{1}&{2}\\
		{2}&{4}
	\end{pmatrix}
	\]
	
	
	\item $\det(A_1|\ldots|A_i|\ldots|A_j|\ldots|A_n) = -\det(A_1|\ldots|A_j|\ldots|A_i|\ldots|A_n)$. То есть если поменять местами два столбца, то определитель изменит знак, например,
	\[
	\det
	\begin{pmatrix}
		{1}&{3}\\
		{2}&{7}
	\end{pmatrix}
	=-
	\det
	\begin{pmatrix}
		{3}&{1}\\
		{7}&{2}
	\end{pmatrix}
	\]
	
	
	\item $\det(A_1|\ldots|A'|\ldots|A'|\ldots|A_n) = 0$, то есть если у матрицы есть два одинаковых столбца, то определитель автоматически равен нулю, например
	\[
	\det
	\begin{pmatrix}
		{1}&{0}\\
		{2}&{0}
	\end{pmatrix} = 0
	\]
	
	\item $\det(A_1|\ldots|\lambda A_i|\ldots|A_n)  = \lambda \det(A_1|\ldots|A_i|\ldots|A_n) $. То есть, если один столбец умножить на одно и то же число, то весь определитель умножится на это число, например,
	\[
	\det
	\begin{pmatrix}
		{1}&{3}\\
		{2}&{9}
	\end{pmatrix}
	=
	3
	\det
	\begin{pmatrix}
		{1}&{1}\\
		{2}&{3}
	\end{pmatrix}
	\]
	
	\item $\det(A_1|\ldots|A_i|\ldots|A_j|\ldots|A_n) = \det(A_1|\ldots|A_i|\ldots|A_j + \lambda A_i|\ldots|A_n)$. То есть, если к одному столбцу матрицы прибавить другой умноженный на коэффициент, то определитель не изменится, например
	\[
	\det
	\begin{pmatrix}
		{1}&{3}\\
		{2}&{7}
	\end{pmatrix}
	=
	\det
	\left(\left.
	\begin{pmatrix}
		{1}\\{2}
	\end{pmatrix}
	\right|
	\begin{pmatrix}
		{3}\\{7}
	\end{pmatrix}
	\right)
	=
	\det
	\left(\left.
	\begin{pmatrix}
		{1}\\{2}
	\end{pmatrix}
	\right|
	\begin{pmatrix}
		{3}\\{7}
	\end{pmatrix}
	+
	2
	\begin{pmatrix}
		{1}\\{2}
	\end{pmatrix}
	\right)
	=
	\det
	\left(\left.
	\begin{pmatrix}
		{1}\\{2}
	\end{pmatrix}
	\right|
	\begin{pmatrix}
		{5}\\{11}
	\end{pmatrix}
	\right)
	=
	\det
	\begin{pmatrix}
		{1}&{5}\\
		{2}&{11}
	\end{pmatrix}
	\]
	
	\item $\det A = \det A^t$. То есть определитель матрицы равен определителю транспонированной матрицы. А значит, все свойства сформулированные выше для столбцов автоматически верны и для строк.
	
	\item Определитель треугольной матрицы
	\[
	\det
	\begin{pmatrix}
		{\lambda_1}&{*}&{\ldots}&{*}\\
		{}&{\lambda_2}&{\ldots}&{*}\\
		{}&{}&{\ddots}&{\vdots}\\
		{}&{}&{}&{\lambda_n}\\
	\end{pmatrix}
	=
	\det
	\begin{pmatrix}
		{\lambda_1}&{}&{}&{}\\
		{*}&{\lambda_2}&{}&{}\\
		{\vdots}&{\vdots}&{\ddots}&{}\\
		{*}&{*}&{\ldots}&{\lambda_n}\\
	\end{pmatrix}
	=
	\lambda_1\lambda_2 \ldots \lambda_n
	\]
	То есть у треугольной матрицы определитель равен произведению ее диагональных элементов. В частности $\det E = 1$ и $\det(\lambda E) = \lambda^n$.
\end{enumerate}

\subsubsection*{Вычисление определителя с помощью элементарных преобразований}

Из сформулированных свойств выше следует, что определитель можно считать так: надо привести матрицу $A$ элементарными преобразованиями к треугольному виду и по пути запоминать некоторые коэффициенты, после чего надо перемножить эти коэффициенты с определителем треугольной матрицы. Давайте продемонстрируем на примере, будем приводить матрицу к ступенчатому виду элементарными преобразованиями строк
\begin{gather*}
	\det
	\begin{pmatrix}
		{0}&{1}&{1}\\
		{-1}&{2}&{1}\\
		{1}&{1}&{-1}\\
	\end{pmatrix}
	=
	-
	\det
	\begin{pmatrix}
		{1}&{1}&{-1}\\
		{-1}&{2}&{1}\\
		{0}&{1}&{1}\\
	\end{pmatrix}
	=
	-\det
	\begin{pmatrix}
		{1}&{1}&{-1}\\
		{0}&{3}&{0}\\
		{0}&{1}&{1}\\
	\end{pmatrix}
	=\\=
	-3
	\det
	\begin{pmatrix}
		{1}&{1}&{-1}\\
		{0}&{1}&{0}\\
		{0}&{1}&{1}\\
	\end{pmatrix}
	=-3
	\det
	\begin{pmatrix}
		{1}&{1}&{-1}\\
		{0}&{1}&{0}\\
		{0}&{0}&{1}\\
	\end{pmatrix}
	=-3
\end{gather*}


\subsubsection*{Связь определителя с произведением}

Для определителя верны следующие формулы
\begin{enumerate}
	\item $\det(AB) = \det(A) \det(B)$
	
	\item $\det(A^{-1}) = \det(A)^{-1}$
	
	\item Пусть $A\in \Matrix{n}$, $C\in \Matrix{m}$ и $B\in \MatrixDim{n}{m}$, тогда
	\[
	\det
	\begin{pmatrix}
		{A}&{B}\\
		{0}&{C}
	\end{pmatrix}
	=
	\det A \det C
	\]
\end{enumerate}

Оказывается, что определитель является единственной функцией $\phi\colon \Matrix{n}\to \mathbb R$ такой, что
\begin{enumerate}
	\item $\phi(AB) = \phi(A) \phi(B)$
	\item 
	$
	\phi
	\begin{pmatrix}
		{1}&{}&{}&{}\\
		{}&{\ddots}&{}&{}\\
		{}&{}&{1}&{}\\
		{}&{}&{}&{\lambda}\\
	\end{pmatrix}
	=\lambda
	$
\end{enumerate}

\subsubsection*{Миноры и алгебраические дополнения}

Пусть $A\in \Matrix{n}$ с элементами $a_{ij}$. Рассмотрим матрицу $D_{ij}\in\Matrix{n-1}$ полученную из $A$ вычеркиванием $i$-ой строки и $j$-го столбца. Определитель матрицы $D_{ij}$ обозначается $M_{ij}$ и называется {\it минором} матрицы $A$ или $ij$-минором для определенности. Число $(-1)^{i+j} M_{ij}$ называется алгебраическим дополнением элемента $a_{ij}$ и обозначается $A_{ij}$.

Покажем как это все выглядит на картинках.  Если мы представим матрицу $A$ в виде
\[
A =
\left(
\begin{array}{c|c|c}
	\cline{2-2}
	{X_{ij}}&{
		\begin{array}{c}
			{*}\\{\vdots}
		\end{array}
	}&{Y_{ij}}\\
	\hline
	\multicolumn{1}{|c|}{
		\begin{array}{cc}
			{*}&{\ldots}
		\end{array}
	}&{a_{ij}}&\multicolumn{1}{c|}{
		\begin{array}{cc}
			{\ldots}&{*}
		\end{array}
	}\\
	\hline
	{Z_{ij}}&{
		\begin{array}{c}
			{\vdots}\\{*}
		\end{array}
	}&{W_{ij}}\\
	\cline{2-2}
\end{array}
\right)
\]
Тогда
\[
D_{ij} =
\begin{pmatrix}
	{X_{ij}}&{Y_{ij}}\\
	{Z_{ij}}&{W_{ij}}\\
\end{pmatrix},\quad
M_{ij} = 
\det
\begin{pmatrix}
	{X_{ij}}&{Y_{ij}}\\
	{Z_{ij}}&{W_{ij}}\\
\end{pmatrix}\quad\text{и}\quad
A_{ij} =
(-1)^{i+j}
\det
\begin{pmatrix}
	{X_{ij}}&{Y_{ij}}\\
	{Z_{ij}}&{W_{ij}}\\
\end{pmatrix}
\]

\subsubsection*{Разложение определителя по строке (столбцу)}


\textbf{Разложение по столбцу}. Для матрицы $A\in \Matrix{n}$ и любого $k$ с условием $1\leqslant k \leqslant n$ верна формула $\det A = \sum_{i = 1} ^ n a_{ik} A_{ik}$. Здесь $A_{ij}$ -- алгебраическое дополнение $a_{ij}$. Например, разложим по второму столбцу
\[
\det\begin{pmatrix}
	{1}&{2}&{3}\\
	{3}&{2}&{1}\\
	{-1}&{0}&{1}\\
\end{pmatrix}=
2 (-1)^{1 + 2} 
\det
\begin{pmatrix}
	{3}&{1}\\
	{-1}&{1}\\
\end{pmatrix}+
2 (-1)^{2+2}
\det
\begin{pmatrix}
	{1}&{3}\\
	{-1}&{1}\\
\end{pmatrix} +
0 (-1)^{3+2}
\det
\begin{pmatrix}
	{1}&{3}\\
	{3}&{1}
\end{pmatrix}
\]

Для разложения по строкам верны ровно те же самые формулы. Их можно получить просто перейдя к транспонированной матрице.
\newpage




\section*{3. Системы линейных алгебраических уравнений. Решение однородной системы. Решение неоднородной системы линейных уравнений. Критерий совместности Кронекера-Капелли}

Общий вид СЛУ и ее однородная версия (ОСЛУ):

\[
\left\{
\begin{aligned}
	a_{11}x_1 + &\ldots + a_{1n}x_n = b_1\\
	&\ldots \\
	a_{m1}x_1 + &\ldots + a_{mn}x_n = b_m
\end{aligned}
\right.\quad\quad
\left\{
\begin{aligned}
	a_{11}x_1 + &\ldots + a_{1n}x_n = 0\\
	&\ldots \\
	a_{m1}x_1 + &\ldots + a_{mn}x_n = 0
\end{aligned}
\right.
\]

\subsection*{Коэффициенты}

Где живут коэффициенты $a_{ij}$ и $b_j$? Варианты:
\begin{itemize}
	\item Вещественные числа $\mathbb R$
	\item Комплексные числа $\mathbb C$
	\item Рациональные числа $\mathbb Q$
\end{itemize}
Для решения СЛУ {\bf НЕ} имеет значения откуда берутся коэффициенты, так как решения будут лежать там же. Потому мы будем работать с числами из $\mathbb R$.

\subsection*{Матрицы связанные со СЛУ}

Для каждой СЛУ введем следующие обозначения:
\[
A= 
\begin{pmatrix}
	a_{11}&\ldots& a_{1n}\\
	\vdots&\ddots&\vdots\\
	a_{m1}& \ldots &a_{mn}
\end{pmatrix}\quad
b = 
\begin{pmatrix}
	b_1\\
	\vdots\\
	b_m
\end{pmatrix} \quad
x =
\begin{pmatrix}
	x_1\\
	\vdots\\
	x_n
\end{pmatrix}\quad
(A|b) =
\left(\left.
\begin{matrix}
	a_{11}&\ldots&a_{1n}\\
	\vdots&\ddots&\vdots\\
	a_{m1}&\ldots&a_{mn}\\
\end{matrix}
\:\right|\:
\begin{matrix}
	b_1\\
	\vdots\\
	b_m\\
\end{matrix}\right)
\]
Названия:
\begin{itemize}
	\item $A$ -- матрица системы
	\item $b$ -- вектор правой части
	\item $(A|b)$ -- расширенная матрица системы
	\item $x$ -- вектор решений
\end{itemize}
Будем кратко записывать СЛУ и ее однородную версию так: $Ax = b$ и $Ax = 0$.
\subsection*{Количество решений}
Случай одного уравнения и одной неизвестной
\begin{itemize}
	\item $x = 0$ -- одно решение
	\item $0x = 0$ -- бесконечное число решений
	\item $0x = 1$ -- нет решений
\end{itemize}


\subsection*{Элементарные преобразования}
\begin{align*}
	\text{I тип: }&
	\left(\left.
	\begin{matrix}
		a_{11}&\ldots&a_{1n}\\
		a_{i1}&\ldots&a_{in}\\
		a_{j1}&\ldots&a_{jn}\\
		a_{m1}&\ldots&a_{mn}
	\end{matrix}
	\:\right|\:
	\begin{matrix}
		b_1\\
		b_i\\
		b_j\\
		b_m
	\end{matrix}
	\right)
	\mapsto
	\left(\left.
	\begin{matrix}
		a_{11}&\ldots&a_{1n}\\
		a_{i1}&\ldots&a_{in}\\
		a_{j1} + \lambda a_{i1}&\ldots&a_{jn}+ \lambda a_{in}\\
		a_{m1}&\ldots&a_{mn}
	\end{matrix}
	\:\right|\:
	\begin{matrix}
		b_1\\
		b_i\\
		b_j+ \lambda b_i\\
		b_m\\
	\end{matrix}
	\right)
	\quad i\neq j
	\\
	\text{II тип: }&
	\left(\left.
	\begin{matrix}
		a_{11}&\ldots&a_{1n}\\
		a_{i1}&\ldots&a_{in}\\
		a_{j1}&\ldots&a_{jn}\\
		a_{m1}&\ldots&a_{mn}
	\end{matrix}
	\:\right|\:
	\begin{matrix}
		b_1\\
		b_i\\
		b_j\\
		b_m
	\end{matrix}
	\right)
	\mapsto
	\left(\left.
	\begin{matrix}
		a_{11}&\ldots&a_{1n}\\
		a_{j1}&\ldots&a_{jn}\\
		a_{i1}&\ldots&a_{in}\\
		a_{m1}&\ldots&a_{mn}
	\end{matrix}
	\:\right|\:
	\begin{matrix}
		b_1\\
		b_j\\
		b_i\\
		b_m
	\end{matrix}
	\right)
	\\
	\text{III тип: }&
	\left(\left.
	\begin{matrix}
		a_{11}&\ldots&a_{1n}\\
		a_{i1}&\ldots&a_{in}\\
		a_{m1}&\ldots&a_{mn}
	\end{matrix}
	\:\right|\:
	\begin{matrix}
		b_1\\
		b_i\\
		b_m
	\end{matrix}
	\right)
	\mapsto
	\left(\left.
	\begin{matrix}
		a_{11}&\ldots&a_{1n}\\
		\lambda a_{i1}&\ldots&\lambda a_{in}\\
		a_{m1}&\ldots&a_{mn}
	\end{matrix}
	\:\right|\:
	\begin{matrix}
		b_1\\
		\lambda b_i\\
		b_m
	\end{matrix}
	\right)
	\quad \lambda \neq 0
\end{align*}
\subsection*{Приведение к ступенчатому виду (алгоритм Гаусса)}
Основной способ решения СЛУ -- привести ее элементарными преобразованиями к простому виду, где множество решений очевидно.\footnote{Данный метод является самым быстрым возможным как для написания программ, так и для ручного вычисления. При вычислениях руками, однако, полезно местами пользоваться <<локальными оптимизациями>>, то есть, если вы видите, что какая-то хитрая комбинация строк сильно упростит вид системы, то сделайте ее.}

Разберем типичный ход алгоритма Гаусса на примере $3$ уравнений и $4$ неизвестных.\footnote{При переходе от одной матрицы к другой я новым коэффициентам даю старые имена, чтобы не захламлять текст новыми обозначениями.}
\subsubsection*{Прямой ход алгоритма Гаусса}
\begin{align*}
	\left(\left.
	\begin{matrix}
		a_{11}& a_{12}&a_{13}& a_{14}\\
		a_{21}& a_{22}&a_{23}& a_{24}\\
		a_{31}& a_{32}&a_{33}& a_{34}\\
	\end{matrix}
	\:\right|\:
	\begin{matrix}
		b_1\\
		b_2\\
		b_3\\
	\end{matrix}
	\right)&
	\quad 2\text{-я строка }-\frac{a_{21}}{a_{11}}\cdot1\text{-я строка}\quad\\
	\left(\left.
	\begin{matrix}
		a_{11}& a_{12}&a_{13}& a_{14}\\
		0& a_{22}&a_{23}& a_{24}\\
		a_{31}& a_{32}&a_{33}& a_{34}\\
	\end{matrix}
	\:\right|\:
	\begin{matrix}
		b_1\\
		b_2\\
		b_3\\
	\end{matrix}
	\right)&
	\quad 3\text{-я строка }-\frac{a_{31}}{a_{11}}\cdot1\text{-я строка}\quad\\
	\left(\left.
	\begin{matrix}
		a_{11}& a_{12}&a_{13}& a_{14}\\
		0& a_{22}&a_{23}& a_{24}\\
		0& a_{32}&a_{33}& a_{34}\\
	\end{matrix}
	\:\right|\:
	\begin{matrix}
		b_1\\
		b_2\\
		b_3\\
	\end{matrix}
	\right)&
	\quad 3\text{-я строка }-\frac{a_{32}}{a_{22}}\cdot2\text{-я строка}\quad\\
	\left(\left.
	\begin{matrix}
		a_{11}& a_{12}&a_{13}& a_{14}\\
		0& a_{22}&a_{23}& a_{24}\\
		0& 0&a_{33}& a_{34}\\
	\end{matrix}
	\:\right|\:
	\begin{matrix}
		b_1\\
		b_2\\
		b_3\\
	\end{matrix}
	\right)&
\end{align*}
В результате данного хода какие-то коэффициенты, например $a_{33}$, могли занулиться, потому возможны следующие принципиально другие случаи\footnote{Это не полный список всех случаев.}
\[
\left(\left.
\begin{matrix}
	\underline{a_{11}}& a_{12}&a_{13}& a_{14}\\
	0& \underline{a_{22}}&a_{23}& a_{24}\\
	0& 0&0& \underline{a_{34}}\\
\end{matrix}
\:\right|\:
\begin{matrix}
	b_1\\
	b_2\\
	b_3\\
\end{matrix}
\right)
\quad
\left(\left.
\begin{matrix}
	\underline{a_{11}}& a_{12}&a_{13}& a_{14}\\
	0& 0&\underline{a_{23}}& a_{24}\\
	0& 0&0& \underline{a_{34}}\\
\end{matrix}
\:\right|\:
\begin{matrix}
	b_1\\
	b_2\\
	b_3\\
\end{matrix}
\right)
\quad
\left(\left.
\begin{matrix}
	\underline{a_{11}}& a_{12}&a_{13}& a_{14}\\
	0& \underline{a_{22}}&a_{23}& a_{24}\\
	0& 0&0& 0\\
\end{matrix}
\:\right|\:
\begin{matrix}
	b_1\\
	b_2\\
	\underline{b_3}\\
\end{matrix}
\right)
\quad
\left(\left.
\begin{matrix}
	\underline{a_{11}}& a_{12}&a_{13}& a_{14}\\
	0& \underline{a_{22}}&a_{23}& a_{24}\\
	0& 0&0& 0\\
\end{matrix}
\:\right|\:
\begin{matrix}
	b_1\\
	b_2\\
	0\\
\end{matrix}
\right)
\]
Подчеркнутые элементы считаются не равными нулю. В ступенчатом виде все переменные (и соответственно коэффициенты перед ними) делятся на главные и неглавные. Главные коэффициенты -- это первые ненулевые коэффициенты в строке (подчеркнутые). Переменные при них называются главными, остальные ненулевые коэффициенты и переменные -- неглавные или свободными.


\subsubsection*{Обратный ход алгоритма Гаусса}
Разберем типичный обратный ход алгоритма Гаусса. Подчеркнутые элементы считаются не равными нулю.
\begin{align*}
	\left(\left.
	\begin{matrix}
		\underline{a_{11}}& a_{12}&a_{13}& a_{14}\\
		0& \underline{a_{22}}&a_{23}& a_{24}\\
		0& 0&\underline{a_{33}}& a_{34}\\
	\end{matrix}
	\:\right|\:
	\begin{matrix}
		b_1\\
		b_2\\
		b_3\\
	\end{matrix}
	\right)&
	\quad \text{разделить }i\text{-ю строку на }a_{ii}\\
	\left(\left.
	\begin{matrix}
		1& a_{12}&a_{13}& a_{14}\\
		0& 1&a_{23}& a_{24}\\
		0& 0&1& a_{34}\\
	\end{matrix}
	\:\right|\:
	\begin{matrix}
		b_1\\
		b_2\\
		b_3\\
	\end{matrix}
	\right)&
	\quad 2\text{-я строка }-a_{23}\cdot3\text{-я строка}\\
	\left(\left.
	\begin{matrix}
		1& a_{12}&a_{13}& a_{14}\\
		0& 1&0& a_{24}\\
		0& 0&1& a_{34}\\
	\end{matrix}
	\:\right|\:
	\begin{matrix}
		b_1\\
		b_2\\
		b_3\\
	\end{matrix}
	\right)&
	\quad 1\text{-я строка }-a_{13}\cdot3\text{-я строка}\\
	\left(\left.
	\begin{matrix}
		1& a_{12}&0& a_{14}\\
		0& 1&0& a_{24}\\
		0& 0&1& a_{34}\\
	\end{matrix}
	\:\right|\:
	\begin{matrix}
		b_1\\
		b_2\\
		b_3\\
	\end{matrix}
	\right)&
	\quad 1\text{-я строка }-a_{12}\cdot2\text{-я строка}\\
	\left(\left.
	\begin{matrix}
		1& 0&0& a_{14}\\
		0& 1&0& a_{24}\\
		0& 0&1& a_{34}\\
	\end{matrix}
	\:\right|\:
	\begin{matrix}
		b_1\\
		b_2\\
		b_3\\
	\end{matrix}
	\right)&
\end{align*}
В специальных случаях приведенных выше, получим
\[
\left(\left.
\begin{matrix}
	1& 0&a_{13}& 0\\
	0& 1&a_{23}& 0\\
	0& 0&0&1\\
\end{matrix}
\:\right|\:
\begin{matrix}
	b_1\\
	b_2\\
	b_3\\
\end{matrix}
\right)
\quad
\left(\left.
\begin{matrix}
	1& a_{12}&0& 0\\
	0& 0&1& 0\\
	0& 0&0&1\\
\end{matrix}
\:\right|\:
\begin{matrix}
	b_1\\
	b_2\\
	b_3\\
\end{matrix}
\right)
\quad
\left(\left.
\begin{matrix}
	1& 0&a_{13}& a_{14}\\
	0&1&a_{23}& a_{24}\\
	0& 0&0& 0\\
\end{matrix}
\:\right|\:
\begin{matrix}
	0\\
	0\\
	1\\
\end{matrix}
\right)
\quad
\left(\left.
\begin{matrix}
	1&0&a_{13}& a_{14}\\
	0& 1&a_{23}& a_{24}\\
	0& 0&0& 0\\
\end{matrix}
\:\right|\:
\begin{matrix}
	b_1\\
	b_2\\
	0\\
\end{matrix}
\right)
\]
Полученный в результате обратного хода вид расширенной матрицы называется улучшенным ступенчатым видом, т.е., это ступенчатый вид, где все коэффициенты при главных неизвестных -- единицы, и все коэффициенты над ними равны нулю.

\subsubsection*{Получение решений}
В системе ниже, выберем переменную $x_4$ как параметр
\[
\left(\left.
\begin{matrix}
	1& 0&0& a_{14}\\
	0& 1&0& a_{24}\\
	0& 0&1& a_{34}\\
\end{matrix}
\:\right|\:
\begin{matrix}
	b_1\\
	b_2\\
	b_3\\
\end{matrix}
\right)
\]
Тогда решения имеют вид\footnote{Операция умножения матрицы на число покомпонентная (умножаем каждый элемент на число). Сумма и разность двух матриц покомпонентная (складываем или вычитаем числа на одних и тех же позициях).}
\[
\begin{pmatrix}
	x_1\\
	x_2\\
	x_3
\end{pmatrix}
=
\begin{pmatrix}
	b_1\\
	b_2\\
	b_3
\end{pmatrix}
-
x_4
\begin{pmatrix}
	a_{14}\\
	a_{24}\\
	a_{34}
\end{pmatrix}
\]
Специальные случаи:
\begin{align*}
	\left(\left.
	\begin{matrix}
		1& 0&a_{13}& 0\\
		0& 1&a_{23}& 0\\
		0& 0&0&1\\
	\end{matrix}
	\:\right|\:
	\begin{matrix}
		b_1\\
		b_2\\
		b_3\\
	\end{matrix}
	\right)&\quad
	\text{Решения:}\quad
	\begin{pmatrix}
		x_1\\
		x_2\\
		x_4
	\end{pmatrix}
	=
	\begin{pmatrix}
		b_1\\
		b_2\\
		b_3
	\end{pmatrix}
	-
	x_3
	\begin{pmatrix}
		a_{13}\\
		a_{23}\\
		0
	\end{pmatrix}\\
	\left(\left.
	\begin{matrix}
		1& a_{12}&0& 0\\
		0& 0&1& 0\\
		0& 0&0&1\\
	\end{matrix}
	\:\right|\:
	\begin{matrix}
		b_1\\
		b_2\\
		b_3\\
	\end{matrix}
	\right)&\quad
	\text{Решения:}\quad
	\begin{pmatrix}
		x_1\\
		x_3\\
		x_4
	\end{pmatrix}
	=
	\begin{pmatrix}
		b_1\\
		b_2\\
		b_3
	\end{pmatrix}
	-
	x_2
	\begin{pmatrix}
		a_{12}\\
		0\\
		0
	\end{pmatrix}\\
	\left(\left.
	\begin{matrix}
		1& 0&a_{13}& a_{14}\\
		0&1&a_{23}& a_{24}\\
		0& 0&0& 0\\
	\end{matrix}
	\:\right|\:
	\begin{matrix}
		0\\
		0\\
		1\\
	\end{matrix}
	\right)&\quad
	\text{Решения:}\quad
	\text{Нет решений, т.к. последнее уравнение }0 = 1
	\\
	\left(\left.
	\begin{matrix}
		1&0&a_{13}& a_{14}\\
		0& 1&a_{23}& a_{24}\\
		0& 0&0& 0\\
	\end{matrix}
	\:\right|\:
	\begin{matrix}
		b_1\\
		b_2\\
		0\\
	\end{matrix}
	\right)&\quad
	\text{Решения:}\quad
	\begin{pmatrix}
		x_1\\
		x_2
	\end{pmatrix}
	=
	\begin{pmatrix}
		b_1\\
		b_2
	\end{pmatrix}
	-
	x_3
	\begin{pmatrix}
		a_{13}\\
		a_{23}
	\end{pmatrix}
	-
	x_4
	\begin{pmatrix}
		a_{14}\\
		a_{24}
	\end{pmatrix}
\end{align*}
\newpage
\end{document}
